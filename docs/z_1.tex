% Options for packages loaded elsewhere
\PassOptionsToPackage{unicode}{hyperref}
\PassOptionsToPackage{hyphens}{url}
\PassOptionsToPackage{dvipsnames,svgnames*,x11names*}{xcolor}
%
\documentclass[
  16pt,
]{krantz}
\usepackage{amsmath,amssymb}
\usepackage{lmodern}
\usepackage{ifxetex,ifluatex}
\ifnum 0\ifxetex 1\fi\ifluatex 1\fi=0 % if pdftex
  \usepackage[T1]{fontenc}
  \usepackage[utf8]{inputenc}
  \usepackage{textcomp} % provide euro and other symbols
\else % if luatex or xetex
  \usepackage{unicode-math}
  \defaultfontfeatures{Scale=MatchLowercase}
  \defaultfontfeatures[\rmfamily]{Ligatures=TeX,Scale=1}
  \setmonofont[Scale=0.7]{Source Code Pro}
\fi
% Use upquote if available, for straight quotes in verbatim environments
\IfFileExists{upquote.sty}{\usepackage{upquote}}{}
\IfFileExists{microtype.sty}{% use microtype if available
  \usepackage[]{microtype}
  \UseMicrotypeSet[protrusion]{basicmath} % disable protrusion for tt fonts
}{}
\makeatletter
\@ifundefined{KOMAClassName}{% if non-KOMA class
  \IfFileExists{parskip.sty}{%
    \usepackage{parskip}
  }{% else
    \setlength{\parindent}{0pt}
    \setlength{\parskip}{6pt plus 2pt minus 1pt}}
}{% if KOMA class
  \KOMAoptions{parskip=half}}
\makeatother
\usepackage{xcolor}
\IfFileExists{xurl.sty}{\usepackage{xurl}}{} % add URL line breaks if available
\IfFileExists{bookmark.sty}{\usepackage{bookmark}}{\usepackage{hyperref}}
\hypersetup{
  pdftitle={Morfología visual},
  pdfauthor={Ricardo Michel MALLQUI BAÑOS},
  colorlinks=true,
  linkcolor=Maroon,
  filecolor=Maroon,
  citecolor=Blue,
  urlcolor=Blue,
  pdfcreator={LaTeX via pandoc}}
\urlstyle{same} % disable monospaced font for URLs
\usepackage{color}
\usepackage{fancyvrb}
\newcommand{\VerbBar}{|}
\newcommand{\VERB}{\Verb[commandchars=\\\{\}]}
\DefineVerbatimEnvironment{Highlighting}{Verbatim}{commandchars=\\\{\}}
% Add ',fontsize=\small' for more characters per line
\usepackage{framed}
\definecolor{shadecolor}{RGB}{248,248,248}
\newenvironment{Shaded}{\begin{snugshade}}{\end{snugshade}}
\newcommand{\AlertTok}[1]{\textcolor[rgb]{0.94,0.16,0.16}{#1}}
\newcommand{\AnnotationTok}[1]{\textcolor[rgb]{0.56,0.35,0.01}{\textbf{\textit{#1}}}}
\newcommand{\AttributeTok}[1]{\textcolor[rgb]{0.77,0.63,0.00}{#1}}
\newcommand{\BaseNTok}[1]{\textcolor[rgb]{0.00,0.00,0.81}{#1}}
\newcommand{\BuiltInTok}[1]{#1}
\newcommand{\CharTok}[1]{\textcolor[rgb]{0.31,0.60,0.02}{#1}}
\newcommand{\CommentTok}[1]{\textcolor[rgb]{0.56,0.35,0.01}{\textit{#1}}}
\newcommand{\CommentVarTok}[1]{\textcolor[rgb]{0.56,0.35,0.01}{\textbf{\textit{#1}}}}
\newcommand{\ConstantTok}[1]{\textcolor[rgb]{0.00,0.00,0.00}{#1}}
\newcommand{\ControlFlowTok}[1]{\textcolor[rgb]{0.13,0.29,0.53}{\textbf{#1}}}
\newcommand{\DataTypeTok}[1]{\textcolor[rgb]{0.13,0.29,0.53}{#1}}
\newcommand{\DecValTok}[1]{\textcolor[rgb]{0.00,0.00,0.81}{#1}}
\newcommand{\DocumentationTok}[1]{\textcolor[rgb]{0.56,0.35,0.01}{\textbf{\textit{#1}}}}
\newcommand{\ErrorTok}[1]{\textcolor[rgb]{0.64,0.00,0.00}{\textbf{#1}}}
\newcommand{\ExtensionTok}[1]{#1}
\newcommand{\FloatTok}[1]{\textcolor[rgb]{0.00,0.00,0.81}{#1}}
\newcommand{\FunctionTok}[1]{\textcolor[rgb]{0.00,0.00,0.00}{#1}}
\newcommand{\ImportTok}[1]{#1}
\newcommand{\InformationTok}[1]{\textcolor[rgb]{0.56,0.35,0.01}{\textbf{\textit{#1}}}}
\newcommand{\KeywordTok}[1]{\textcolor[rgb]{0.13,0.29,0.53}{\textbf{#1}}}
\newcommand{\NormalTok}[1]{#1}
\newcommand{\OperatorTok}[1]{\textcolor[rgb]{0.81,0.36,0.00}{\textbf{#1}}}
\newcommand{\OtherTok}[1]{\textcolor[rgb]{0.56,0.35,0.01}{#1}}
\newcommand{\PreprocessorTok}[1]{\textcolor[rgb]{0.56,0.35,0.01}{\textit{#1}}}
\newcommand{\RegionMarkerTok}[1]{#1}
\newcommand{\SpecialCharTok}[1]{\textcolor[rgb]{0.00,0.00,0.00}{#1}}
\newcommand{\SpecialStringTok}[1]{\textcolor[rgb]{0.31,0.60,0.02}{#1}}
\newcommand{\StringTok}[1]{\textcolor[rgb]{0.31,0.60,0.02}{#1}}
\newcommand{\VariableTok}[1]{\textcolor[rgb]{0.00,0.00,0.00}{#1}}
\newcommand{\VerbatimStringTok}[1]{\textcolor[rgb]{0.31,0.60,0.02}{#1}}
\newcommand{\WarningTok}[1]{\textcolor[rgb]{0.56,0.35,0.01}{\textbf{\textit{#1}}}}
\usepackage{longtable,booktabs,array}
\usepackage{calc} % for calculating minipage widths
% Correct order of tables after \paragraph or \subparagraph
\usepackage{etoolbox}
\makeatletter
\patchcmd\longtable{\par}{\if@noskipsec\mbox{}\fi\par}{}{}
\makeatother
% Allow footnotes in longtable head/foot
\IfFileExists{footnotehyper.sty}{\usepackage{footnotehyper}}{\usepackage{footnote}}
\makesavenoteenv{longtable}
\setlength{\emergencystretch}{3em} % prevent overfull lines
\providecommand{\tightlist}{%
  \setlength{\itemsep}{0pt}\setlength{\parskip}{0pt}}
\setcounter{secnumdepth}{5}
\usepackage[spanish,es-lcroman]{babel}
\usepackage{booktabs}
\usepackage{graphicx}
\usepackage{amsmath}
\usepackage{makeidx}
\makeindex
%\usepackage{showframe}
%\usepackage[a4paper]{geometry}
%\geometry{verbose,tmargin=3cm,bmargin=3cm,lmargin=3.5cm,rmargin=3cm}

\usepackage{times}
\renewcommand{\rmdefault}{ptm}
%\usepackage[lite,subscriptcorrection,nofontinfo,zswash]{mtpro2}

\usepackage{graphicx}

% Determine if the image is too wide for the page.
\makeatletter
\def\ScaleIfNeeded{%
  \ifdim\Gin@nat@width>\linewidth
    \linewidth
  \else
    \Gin@nat@width
  \fi
}
\makeatother

% Resize figures that are too wide for the page.
\let\oldincludegraphics\includegraphics
\renewcommand\includegraphics[2][]{%
  \oldincludegraphics[scale=0.85]{#2}
}

\usepackage{amsthm}
\makeatletter
\def\thm@space@setup{%
  \thm@preskip=8pt plus 2pt minus 4pt
  \thm@postskip=\thm@preskip
}
\makeatother



\flushbottom 

\frontmatter

\ifluatex
  \usepackage{selnolig}  % disable illegal ligatures
\fi
\usepackage[]{natbib}
\bibliographystyle{apalike}

\title{Morfología visual}
\author{Ricardo Michel MALLQUI BAÑOS}
\date{2021-05-25}

\usepackage{amsthm}
\newtheorem{theorem}{Teorema}[chapter]
\newtheorem{lemma}{Lema}[chapter]
\newtheorem{corollary}{Corolario}[chapter]
\newtheorem{proposition}{Proposición}[chapter]
\newtheorem{conjecture}{Conjectura}[chapter]
\theoremstyle{definition}
\newtheorem{definition}{Definición}[chapter]
\theoremstyle{definition}
\newtheorem{example}{Ejemplo}[chapter]
\theoremstyle{definition}
\newtheorem{exercise}{Ejercicio}[chapter]
\theoremstyle{definition}
\newtheorem{hypothesis}{Hypothesis}[chapter]
\theoremstyle{remark}
\newtheorem*{remark}{Observación}
\newtheorem*{solution}{Solución}
\begin{document}
\maketitle

%\cleardoublepage\newpage\thispagestyle{empty}\null
%\cleardoublepage\newpage\thispagestyle{empty}\null
%\cleardoublepage\newpage
\thispagestyle{empty}
\begin{center}
\includegraphics{U.pdf}
\end{center}

%\setlength{\abovedisplayskip}{-5pt}
%\setlength{\abovedisplayshortskip}{-5pt}

{
\hypersetup{linkcolor=}
\setcounter{tocdepth}{2}
\tableofcontents
}
\listoftables
\listoffigures
\newcommand{\N}{\mathbb{N}}
\newcommand{\R}{\mathbb{R}}
\newcommand{\CC}{\mathbb{C}}
\newcommand{\I}{\mathbb{I}}
\newcommand{\f}{\mathbb{f}}
\newcommand{\X}{\mathbb{X}}
\newcommand{\D}{\mathbb{D}}
\newcommand{\Z}{\mathbb{Z}}
\newcommand{\Q}{\mathbb{Q}}
\newcommand{\norm}[1]{\left\Vert#1\right\Vert}
\newcommand{\abs}[1]{\left\vert#1\right\vert}
\newcommand{\set}[1]{\left\{#1\right\}}
\newcommand{\seq}[1]{\left<#1\right>}
\newcommand{\co}[1]{\left[#1\right]}
\newcommand{\cc}[1]{\left(#1\right)}
\newcommand{\J}{\mathcal{J}}
\newcommand{\K}{\mathcal{K}}
\newcommand{\M}{\mathcal{M}}
\newcommand{\F}{\mathcal{F}}

\hypertarget{resumen}{%
\chapter*{Resumen}\label{resumen}}


La importancia del estudio de la forma en el arte plástico es explicita debido a la manipulación de estas en el espacio bidimensional y tridimensional, en el plano bidimensional se estudian aspectos geometricos partiendo desde la forma de un punto hasta formas orgánicas con comportamientos similares al de los conjuntos fractales de Mandelbrot y Julia y en el espacio tridimensional se realizara un estudio sobre formas que viven en este espacio es decir tanto las formas bidimensionales además de los sólidos geométricos y formas organicas como las superficies organicas como los conjuntos de Mandelbrots 3D y julia 3D. Finalmente se realiza composiciones con estas formas, utilizando principios compositivos con el objetivo de reunir los conocimientos previos y reconocer la utilidad de su estudio previo. Luego se reconocerán estas formas como contenedores de formas existentes en la naturaleza tales como la fitomorfología, la zoomorfología, la geomorfologia

\hypertarget{introducciuxf3n}{%
\chapter*{Introducción}\label{introducciuxf3n}}


El estudio de la forma de manera aislada o compositiva, en el arte plástico es de importancia para la manipulación correcta de estas, generando y representación lógica; es decir desprovista de la intuición, la intuición, generalmente distorsiona el aspecto verdadero de las formas.

El espacio bidimensional se denota con el símbolo \(\mathbb{R}^2\) y al espacio tridimensional se denota con el símbolo \(\mathbb{R}^3\). Las formas bidimensionales pueden existir tanto el el espacio bidimensional y tridimensional y Las formas tridimensionales unicamente pueden existir el el espacio tridimensional.

Las formas que reunen todas las caracteristicas de las formas bidimnesionales y tridmensionales son los las formas llamadas fractales. que se generana bajo los procesos de iteración o recursividad de ciertas formas básicas llamadas módulos, exiten modelos secuenciales que indican la recursividad, la formas concernientes a los fractales generadas por los números complejos son las llamadas ocacionalmente conjuntos de Mandelbrot y Julia en el espacio bidimensional y el el espacio tridmencional suelen llamarse conjuntos de Mandelbrot 3D y Julia 3D, estas formas manifiestan una variedad infinita

El libro se realiza bajo las teorias y conceptos de diversas áreas tales como la botánica la geometría descriptiva, modelos matemáticos, que proveen de conceptos y objetos utilizados hasta el momento de manera intuitiva en las artes plásticas es decir se interrrelaciona estos conocimientos con el objetivos de agruparlos y demostrar que este agrupamiento de manera lógica y secuencial, esta avalada evidentemente por las teorías y la intuición genuina.

Las formas orgánicas son aquellas escapan a una geometría especifica ya estudiadas pero pueden ser estudiadas ajo criterios de geometrización es decir generando una red geométrica, esto es identificando los puntos sobre la forma que sirvan de anclaje o vertice, estos puntos pueden pertenecer o no la forma en el primer caso se debe considerar que sean puntos más resaltantes de la forma, en el sugundo caso deben ser punto de manera que generan segmentos tangentes a las formas.

El libro se compone de cinco capítulos en los cuales se describen los temas de manera secuencial además de dos apéndices que sirven como reforzamiento de la ideas vertidas en el texto es decir en el primer capitulo se describe la teoría de la forma en el espacio bidimensional, en el capitulo 2 se describe la teoría de las formas tridimensionales, en el tercer capítulo concierne a la teoría de formas compositivas, el capítulo 4 formas orgánicas y su geometría, el capítulo 5 formas abstractas y su geometria

\mainmatter

\hypertarget{formas-bidimesionales}{%
\chapter{Formas bidimesionales}\label{formas-bidimesionales}}

En este capítulo se describirán definiciones y conceptos sobre las formas básicas que existen el espacio bidimensional \(\mathbb{R}\) tales como el punto, la linea, los polígonos, las formas orgánicas y los fractales bidimensionales

\hypertarget{el-punto-y-la-linea}{%
\section{El punto y la linea}\label{el-punto-y-la-linea}}

\begin{definition}[Espacio bidimesional]
\protect\hypertarget{def:r2}{}{\label{def:r2} \iffalse (Espacio bidimesional) \fi{} }El espacio bidimensional es un conjunto de puntos es un ente grafico, considerada como la mínima unidad en la representacion o el diseño.
\end{definition}

\begin{definition}[El punto]
\protect\hypertarget{def:punto}{}{\label{def:punto} \iffalse (El punto) \fi{} }El punto es un ente grafico, considerada como la mínima unidad en la representacion o el diseño.
\end{definition}

La ubicación de un punto se realiza con la ayuda de un sistema de ejes coordenados refiérase a la Definición \ref{def:r2} compuestas de dos ejes el eje \(x\) (eje de las abscisas) y el eje \(y\) (eje de las ordenadas), ademas de una etiqueta de ser necesaria, es decir el punto \(W=(x,y)\) indica que esta ubicada a \(x\) unidades sobre el eje \(x\), del centro del sistema de ejes coordenados, a la derecha si \(x\) es positivo y a la izquierda si \(x\) es negativo; y \(y\) unidades sobre el eje \(y\), del centro del sistema de ejes coordenados, a la arriba si \(x\) es positivo y a la abajo si \(x\) es negativo.

\begin{definition}[La linea]
\protect\hypertarget{def:linea}{}{\label{def:linea} \iffalse (La linea) \fi{} }La linea considerada como el conjunto de puntos distribuidas de manera secuencial es decir yuxtapuestas.
\end{definition}

\hypertarget{poluxedgonales}{%
\section{Polígonales}\label{poluxedgonales}}

\hypertarget{curvas-cerradas}{%
\section{Curvas cerradas}\label{curvas-cerradas}}

\hypertarget{la-circunferencia}{%
\subsection{La circunferencia}\label{la-circunferencia}}

\begin{definition}[La circunferencia]
\protect\hypertarget{def:circulo}{}{\label{def:circulo} \iffalse (La circunferencia) \fi{} }Es la curva geenrada por los puntos que equidistan de un punto llamado centro de la cirunferencia
\end{definition}

\hypertarget{la-elipse}{%
\subsection{La elipse}\label{la-elipse}}

\hypertarget{trasformaciuxf3n-de-la-elipse}{%
\subsection{Trasformación de la elipse}\label{trasformaciuxf3n-de-la-elipse}}

\hypertarget{lugares-geomuxe9tricos}{%
\section{Lugares geométricos}\label{lugares-geomuxe9tricos}}

\hypertarget{las-cuxf3nicas}{%
\subsection{Las cónicas}\label{las-cuxf3nicas}}

\hypertarget{otros}{%
\subsection{Otros\ldots{}}\label{otros}}

\hypertarget{fractales-bidimesionales}{%
\section{Fractales bidimesionales}\label{fractales-bidimesionales}}

\hypertarget{fractales-cluxe1sicos}{%
\subsection{Fractales clásicos}\label{fractales-cluxe1sicos}}

\begin{definition}[Triangulos de Sierpinski]
\protect\hypertarget{def:sierpinski}{}{\label{def:sierpinski} \iffalse (Triangulos de Sierpinski) \fi{} }wwwwwwwwwwwwwwwwwwwwwwwwwww
\end{definition}

\begin{definition}[Copo de nieve de Kosh]
\protect\hypertarget{def:kosh}{}{\label{def:kosh} \iffalse (Copo de nieve de Kosh) \fi{} }wwwwwwwwwwwwwwwwwwwwwwwwwww
\end{definition}

\begin{definition}[Arbol fractal]
\protect\hypertarget{def:arbol}{}{\label{def:arbol} \iffalse (Arbol fractal) \fi{} }wwwwwwwwwwwwwwwwwwwwwwwwwww
\end{definition}

\hypertarget{formas-tridimesionales}{%
\chapter{Formas tridimesionales}\label{formas-tridimesionales}}

wwwwwwwwwwwwwwwwwwwwwwwwwwl plano Here is a review of existing methods.

\hypertarget{superficies-poliedricas}{%
\section{Superficies poliedricas}\label{superficies-poliedricas}}

www

\hypertarget{solidos-platuxf3nicos}{%
\subsection{Solidos platónicos}\label{solidos-platuxf3nicos}}

\hypertarget{los-prismas}{%
\subsection{Los prismas}\label{los-prismas}}

\hypertarget{superficies-de-revolucion-y-regladas}{%
\section{Superficies de revolucion y regladas}\label{superficies-de-revolucion-y-regladas}}

\hypertarget{superficies-curvas}{%
\section{Superficies curvas}\label{superficies-curvas}}

\hypertarget{cerradas}{%
\subsection{Cerradas}\label{cerradas}}

Esfera Elipsoide

\hypertarget{abiertas}{%
\subsection{Abiertas}\label{abiertas}}

\hypertarget{orientables}{%
\subsection{Orientables}\label{orientables}}

\hypertarget{no-orientables}{%
\subsection{No orientables}\label{no-orientables}}

\hypertarget{fractales-3d}{%
\section{Fractales 3D}\label{fractales-3d}}

\hypertarget{composiciuxf3n-de-formas}{%
\chapter{Composición de formas}\label{composiciuxf3n-de-formas}}

\hypertarget{operaciones-con-formas}{%
\section{Operaciones con formas}\label{operaciones-con-formas}}

\hypertarget{union}{%
\subsection{Union}\label{union}}

\begin{definition}
\protect\hypertarget{def:unnamed-chunk-1}{}{\label{def:unnamed-chunk-1} }wwwwwwwwwwwwwwwwwwwwwww
\end{definition}

\hypertarget{interseccion}{%
\subsection{Interseccion}\label{interseccion}}

\hypertarget{diferencia}{%
\subsection{Diferencia}\label{diferencia}}

\hypertarget{diferencia-simetrica}{%
\subsection{Diferencia simetrica}\label{diferencia-simetrica}}

\hypertarget{complemento}{%
\subsection{Complemento}\label{complemento}}

\hypertarget{componiendo-escenas}{%
\section{Componiendo escenas}\label{componiendo-escenas}}

\hypertarget{utilizando-software}{%
\subsection{Utilizando software}\label{utilizando-software}}

\hypertarget{el-bodegon}{%
\subsection{El bodegon}\label{el-bodegon}}

\hypertarget{la-superficie}{%
\subsection{La superficie}\label{la-superficie}}

\hypertarget{wwwwwwwwwww}{%
\subsection{wwwwwwwwwww}\label{wwwwwwwwwww}}

\hypertarget{formas-organicas}{%
\chapter{Formas organicas}\label{formas-organicas}}

\hypertarget{geometrizacion}{%
\section{Geometrizacion}\label{geometrizacion}}

\hypertarget{redes}{%
\section{Redes}\label{redes}}

\hypertarget{geomorfologia}{%
\section{Geomorfologia}\label{geomorfologia}}

\hypertarget{fitomorfologia}{%
\section{Fitomorfologia}\label{fitomorfologia}}

\hypertarget{zoomorfologia}{%
\section{Zoomorfologia}\label{zoomorfologia}}

\hypertarget{formas-abstractas}{%
\chapter{Formas abstractas}\label{formas-abstractas}}

\hypertarget{caos-y-orden}{%
\section{Caos y orden}\label{caos-y-orden}}

\citep{bookdown2016}wwwwwww \citep{vincze2014college}

\hypertarget{ejercicios}{%
\section{Ejercicios}\label{ejercicios}}

wwwwwwwwwwwwwwwwwwwwwwwwwwwwwwwwwwww

\hypertarget{formas-matematicas}{%
\chapter{Formas Matematicas}\label{formas-matematicas}}

\begin{Shaded}
\begin{Highlighting}[]
\NormalTok{knitr}\SpecialCharTok{::}\FunctionTok{kable}\NormalTok{(}
  \FunctionTok{head}\NormalTok{(iris, }\DecValTok{20}\NormalTok{), }\AttributeTok{caption =} \StringTok{\textquotesingle{}Here is a nice table!\textquotesingle{}}\NormalTok{,}
  \AttributeTok{booktabs =} \ConstantTok{TRUE}
\NormalTok{)}
\end{Highlighting}
\end{Shaded}

\begin{table}

\caption{\label{tab:nice-tab}Here is a nice table!}
\centering
\begin{tabular}[t]{rrrrl}
\toprule
Sepal.Length & Sepal.Width & Petal.Length & Petal.Width & Species\\
\midrule
5.1 & 3.5 & 1.4 & 0.2 & setosa\\
4.9 & 3.0 & 1.4 & 0.2 & setosa\\
4.7 & 3.2 & 1.3 & 0.2 & setosa\\
4.6 & 3.1 & 1.5 & 0.2 & setosa\\
5.0 & 3.6 & 1.4 & 0.2 & setosa\\
\addlinespace
5.4 & 3.9 & 1.7 & 0.4 & setosa\\
4.6 & 3.4 & 1.4 & 0.3 & setosa\\
5.0 & 3.4 & 1.5 & 0.2 & setosa\\
4.4 & 2.9 & 1.4 & 0.2 & setosa\\
4.9 & 3.1 & 1.5 & 0.1 & setosa\\
\addlinespace
5.4 & 3.7 & 1.5 & 0.2 & setosa\\
4.8 & 3.4 & 1.6 & 0.2 & setosa\\
4.8 & 3.0 & 1.4 & 0.1 & setosa\\
4.3 & 3.0 & 1.1 & 0.1 & setosa\\
5.8 & 4.0 & 1.2 & 0.2 & setosa\\
\addlinespace
5.7 & 4.4 & 1.5 & 0.4 & setosa\\
5.4 & 3.9 & 1.3 & 0.4 & setosa\\
5.1 & 3.5 & 1.4 & 0.3 & setosa\\
5.7 & 3.8 & 1.7 & 0.3 & setosa\\
5.1 & 3.8 & 1.5 & 0.3 & setosa\\
\bottomrule
\end{tabular}
\end{table}

wwwwwwwwwwwwwwwwwwwwwwwwwww

\hypertarget{funciones}{%
\section{Funciones}\label{funciones}}

wwwwwwwwww\index{wwwwwwww} \citep{vincze2014college}

\hypertarget{ejercicios-1}{%
\section{Ejercicios}\label{ejercicios-1}}

\hypertarget{appendix-apendice}{%
\appendix \addcontentsline{toc}{chapter}{\appendixname}}


Temas de reforzamiento o conocimientos preliminares que son necesarias para entender el contenido.

\hypertarget{trasformaciones}{%
\chapter{Trasformaciones}\label{trasformaciones}}

Se refiere a las transformaciones o modificaciones que pueden sufrir las formas, es decir los achatamientos, las elongaciones los cambios de posición etc., mediante la manipulación de los puntos pertenecientes a la forma.

\begin{definition}[Transformación]
\protect\hypertarget{def:transformacion}{}{\label{def:transformacion} \iffalse (Transformación) \fi{} }Una transformacion es el proceso de modificar una forma covirtiendola en otra
\end{definition}

\hypertarget{trasformaciones-elementales}{%
\section{Trasformaciones elementales}\label{trasformaciones-elementales}}

En esta seccion se trata sobre la trasformaciones basicas que son la traslación, la rotación, la reflexión y la homotescia o escala

\hypertarget{traslacion}{%
\subsection{Traslacion}\label{traslacion}}

\begin{definition}[Traslación]
\protect\hypertarget{def:traslacion}{}{\label{def:traslacion} \iffalse (Traslación) \fi{} }La traslacion de un objeto, consiste en mover todos los puntos del objeto en el espacio 2D o 3D en una solo dirección, un solo sentido y a una distancia determinada.
\end{definition}

\begin{example}
\protect\hypertarget{exm:unnamed-chunk-2}{}{\label{exm:unnamed-chunk-2} }Sea figura \texttt{\ref{fig:Doge}} la derección de \(37^\circ\), el sentido indicada por la flecha y la distancia 5 unidades.
\end{example}

www

\begin{figure}

{\centering \includegraphics{elipse} 

}

\caption{Hola}\label{fig:Doge}
\end{figure}

\begin{example}
\protect\hypertarget{exm:unnamed-chunk-3}{}{\label{exm:unnamed-chunk-3} }www wwwwwwwwwwww wwwwwwwwwwwwwww
\end{example}

En la escala u homotescia tambien existen procedimientos de proporción \ref{fig:Doge}

\hypertarget{rotacion}{%
\subsection{Rotacion}\label{rotacion}}

\begin{definition}[Traslación]
\protect\hypertarget{def:rotacion}{}{\label{def:rotacion} \iffalse (Traslación) \fi{} }La traslacion es el proceso de mover todos los puntos de un objeto en el espacio 2D o 3D en una solo dirección y sentido a una distancia determinada
\end{definition}

\hypertarget{reflexiuxf3n}{%
\subsection{Reflexión}\label{reflexiuxf3n}}

La traslacion es el proceso de mover todos los puntos de un objeto en el espacio 2D o 3D en una solo dirección y sentido a una distancia determinada

\hypertarget{homotescia}{%
\subsection{Homotescia}\label{homotescia}}

La traslacion es el proceso de mover todos los puntos de un objeto en el espacio 2D o 3D en una solo dirección y sentido a una distancia determinada

\hypertarget{trasformaciones-topoluxf3gicas}{%
\section{Trasformaciones topológicas}\label{trasformaciones-topoluxf3gicas}}

La traslacion es el proceso de mover todos los puntos de un objeto en el espacio 2D o 3D en una solo dirección y sentido a una distancia determinada

\hypertarget{homeomofismo}{%
\subsection{Homeomofismo}\label{homeomofismo}}

La traslacion es el proceso de mover todos los puntos de un objeto en el espacio 2D o 3D en una solo dirección y sentido a una distancia determinada

\hypertarget{homomorfismo}{%
\subsection{Homomorfismo}\label{homomorfismo}}

La traslacion es el proceso de mover todos los puntos de un objeto en el espacio 2D o 3D en una solo dirección y sentido a una distancia determinada \citep{xie2015}

\hypertarget{isomorfismo}{%
\subsection{Isomorfismo}\label{isomorfismo}}

\hypertarget{wwwwwomorfismo}{%
\subsection{wwwwwomorfismo}\label{wwwwwomorfismo}}

\hypertarget{centro-de-masa}{%
\chapter{Centro de masa}\label{centro-de-masa}}

\hypertarget{centro-de-masa-de-objetos-2d}{%
\section{Centro de masa de objetos 2D}\label{centro-de-masa-de-objetos-2d}}

\hypertarget{metodos-matematicos}{%
\subsection{Metodos matematicos}\label{metodos-matematicos}}

\hypertarget{metodos-tecnicos}{%
\subsection{Metodos tecnicos}\label{metodos-tecnicos}}

\hypertarget{muxe9todo-del-borde-de-la-mesa}{%
\subsubsection{Método del borde de la mesa}\label{muxe9todo-del-borde-de-la-mesa}}

\hypertarget{muxe9todo-de-la-plomada}{%
\subsubsection{Método de la plomada}\label{muxe9todo-de-la-plomada}}

\hypertarget{centro-de-masa-de-objetos-3d}{%
\section{Centro de masa de objetos 3D}\label{centro-de-masa-de-objetos-3d}}

\hypertarget{metodos-matematicos-1}{%
\subsection{Metodos matematicos}\label{metodos-matematicos-1}}

\hypertarget{metodos-tecnicos-1}{%
\subsection{Metodos tecnicos}\label{metodos-tecnicos-1}}

\hypertarget{muxe9todo-de-las-secciones}{%
\subsubsection{Método de las secciones}\label{muxe9todo-de-las-secciones}}

\hypertarget{muxe9todo-de-la-plomada-1}{%
\subsubsection{Método de la plomada}\label{muxe9todo-de-la-plomada-1}}

  \bibliography{book.bib,packages.bib}

\printindex

\end{document}
